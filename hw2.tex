% Options for packages loaded elsewhere
\PassOptionsToPackage{unicode}{hyperref}
\PassOptionsToPackage{hyphens}{url}
%
\documentclass[
]{article}
\usepackage{amsmath,amssymb}
\usepackage{lmodern}
\usepackage{iftex}
\ifPDFTeX
  \usepackage[T1]{fontenc}
  \usepackage[utf8]{inputenc}
  \usepackage{textcomp} % provide euro and other symbols
\else % if luatex or xetex
  \usepackage{unicode-math}
  \defaultfontfeatures{Scale=MatchLowercase}
  \defaultfontfeatures[\rmfamily]{Ligatures=TeX,Scale=1}
\fi
% Use upquote if available, for straight quotes in verbatim environments
\IfFileExists{upquote.sty}{\usepackage{upquote}}{}
\IfFileExists{microtype.sty}{% use microtype if available
  \usepackage[]{microtype}
  \UseMicrotypeSet[protrusion]{basicmath} % disable protrusion for tt fonts
}{}
\makeatletter
\@ifundefined{KOMAClassName}{% if non-KOMA class
  \IfFileExists{parskip.sty}{%
    \usepackage{parskip}
  }{% else
    \setlength{\parindent}{0pt}
    \setlength{\parskip}{6pt plus 2pt minus 1pt}}
}{% if KOMA class
  \KOMAoptions{parskip=half}}
\makeatother
\usepackage{xcolor}
\usepackage[margin=1in]{geometry}
\usepackage{color}
\usepackage{fancyvrb}
\newcommand{\VerbBar}{|}
\newcommand{\VERB}{\Verb[commandchars=\\\{\}]}
\DefineVerbatimEnvironment{Highlighting}{Verbatim}{commandchars=\\\{\}}
% Add ',fontsize=\small' for more characters per line
\usepackage{framed}
\definecolor{shadecolor}{RGB}{248,248,248}
\newenvironment{Shaded}{\begin{snugshade}}{\end{snugshade}}
\newcommand{\AlertTok}[1]{\textcolor[rgb]{0.94,0.16,0.16}{#1}}
\newcommand{\AnnotationTok}[1]{\textcolor[rgb]{0.56,0.35,0.01}{\textbf{\textit{#1}}}}
\newcommand{\AttributeTok}[1]{\textcolor[rgb]{0.77,0.63,0.00}{#1}}
\newcommand{\BaseNTok}[1]{\textcolor[rgb]{0.00,0.00,0.81}{#1}}
\newcommand{\BuiltInTok}[1]{#1}
\newcommand{\CharTok}[1]{\textcolor[rgb]{0.31,0.60,0.02}{#1}}
\newcommand{\CommentTok}[1]{\textcolor[rgb]{0.56,0.35,0.01}{\textit{#1}}}
\newcommand{\CommentVarTok}[1]{\textcolor[rgb]{0.56,0.35,0.01}{\textbf{\textit{#1}}}}
\newcommand{\ConstantTok}[1]{\textcolor[rgb]{0.00,0.00,0.00}{#1}}
\newcommand{\ControlFlowTok}[1]{\textcolor[rgb]{0.13,0.29,0.53}{\textbf{#1}}}
\newcommand{\DataTypeTok}[1]{\textcolor[rgb]{0.13,0.29,0.53}{#1}}
\newcommand{\DecValTok}[1]{\textcolor[rgb]{0.00,0.00,0.81}{#1}}
\newcommand{\DocumentationTok}[1]{\textcolor[rgb]{0.56,0.35,0.01}{\textbf{\textit{#1}}}}
\newcommand{\ErrorTok}[1]{\textcolor[rgb]{0.64,0.00,0.00}{\textbf{#1}}}
\newcommand{\ExtensionTok}[1]{#1}
\newcommand{\FloatTok}[1]{\textcolor[rgb]{0.00,0.00,0.81}{#1}}
\newcommand{\FunctionTok}[1]{\textcolor[rgb]{0.00,0.00,0.00}{#1}}
\newcommand{\ImportTok}[1]{#1}
\newcommand{\InformationTok}[1]{\textcolor[rgb]{0.56,0.35,0.01}{\textbf{\textit{#1}}}}
\newcommand{\KeywordTok}[1]{\textcolor[rgb]{0.13,0.29,0.53}{\textbf{#1}}}
\newcommand{\NormalTok}[1]{#1}
\newcommand{\OperatorTok}[1]{\textcolor[rgb]{0.81,0.36,0.00}{\textbf{#1}}}
\newcommand{\OtherTok}[1]{\textcolor[rgb]{0.56,0.35,0.01}{#1}}
\newcommand{\PreprocessorTok}[1]{\textcolor[rgb]{0.56,0.35,0.01}{\textit{#1}}}
\newcommand{\RegionMarkerTok}[1]{#1}
\newcommand{\SpecialCharTok}[1]{\textcolor[rgb]{0.00,0.00,0.00}{#1}}
\newcommand{\SpecialStringTok}[1]{\textcolor[rgb]{0.31,0.60,0.02}{#1}}
\newcommand{\StringTok}[1]{\textcolor[rgb]{0.31,0.60,0.02}{#1}}
\newcommand{\VariableTok}[1]{\textcolor[rgb]{0.00,0.00,0.00}{#1}}
\newcommand{\VerbatimStringTok}[1]{\textcolor[rgb]{0.31,0.60,0.02}{#1}}
\newcommand{\WarningTok}[1]{\textcolor[rgb]{0.56,0.35,0.01}{\textbf{\textit{#1}}}}
\usepackage{graphicx}
\makeatletter
\def\maxwidth{\ifdim\Gin@nat@width>\linewidth\linewidth\else\Gin@nat@width\fi}
\def\maxheight{\ifdim\Gin@nat@height>\textheight\textheight\else\Gin@nat@height\fi}
\makeatother
% Scale images if necessary, so that they will not overflow the page
% margins by default, and it is still possible to overwrite the defaults
% using explicit options in \includegraphics[width, height, ...]{}
\setkeys{Gin}{width=\maxwidth,height=\maxheight,keepaspectratio}
% Set default figure placement to htbp
\makeatletter
\def\fps@figure{htbp}
\makeatother
\setlength{\emergencystretch}{3em} % prevent overfull lines
\providecommand{\tightlist}{%
  \setlength{\itemsep}{0pt}\setlength{\parskip}{0pt}}
\setcounter{secnumdepth}{-\maxdimen} % remove section numbering
\ifLuaTeX
  \usepackage{selnolig}  % disable illegal ligatures
\fi
\IfFileExists{bookmark.sty}{\usepackage{bookmark}}{\usepackage{hyperref}}
\IfFileExists{xurl.sty}{\usepackage{xurl}}{} % add URL line breaks if available
\urlstyle{same} % disable monospaced font for URLs
\hypersetup{
  pdftitle={hw2},
  hidelinks,
  pdfcreator={LaTeX via pandoc}}

\title{hw2}
\author{}
\date{\vspace{-2.5em}2022-10-06}

\begin{document}
\maketitle

Set up:

\begin{verbatim}
## -- Attaching packages -------------------------------------- tidymodels 1.0.0 --
\end{verbatim}

\begin{verbatim}
## v broom        1.0.1      v recipes      1.0.1 
## v dials        1.0.0      v rsample      1.1.0 
## v dplyr        1.0.10     v tibble       3.1.8 
## v ggplot2      3.3.6      v tidyr        1.2.1 
## v infer        1.0.3      v tune         1.0.0 
## v modeldata    1.0.1      v workflows    1.1.0 
## v parsnip      1.0.1      v workflowsets 1.0.0 
## v purrr        0.3.4      v yardstick    1.1.0
\end{verbatim}

\begin{verbatim}
## -- Conflicts ----------------------------------------- tidymodels_conflicts() --
## x purrr::discard() masks scales::discard()
## x dplyr::filter()  masks stats::filter()
## x dplyr::lag()     masks stats::lag()
## x recipes::step()  masks stats::step()
## * Use suppressPackageStartupMessages() to eliminate package startup messages
\end{verbatim}

\begin{verbatim}
## -- Attaching packages --------------------------------------- tidyverse 1.3.2 --
## v readr   2.1.2     v forcats 0.5.2
## v stringr 1.4.1     
## -- Conflicts ------------------------------------------ tidyverse_conflicts() --
## x readr::col_factor() masks scales::col_factor()
## x purrr::discard()    masks scales::discard()
## x dplyr::filter()     masks stats::filter()
## x stringr::fixed()    masks recipes::fixed()
## x dplyr::lag()        masks stats::lag()
## x readr::spec()       masks yardstick::spec()
## Rows: 4177 Columns: 9
## -- Column specification --------------------------------------------------------
## Delimiter: ","
## chr (1): type
## dbl (8): longest_shell, diameter, height, whole_weight, shucked_weight, visc...
## 
## i Use `spec()` to retrieve the full column specification for this data.
## i Specify the column types or set `show_col_types = FALSE` to quiet this message.
\end{verbatim}

\begin{verbatim}
## # A tibble: 4,177 x 9
##    type  longest_shell diameter height whole_wei~1 shuck~2 visce~3 shell~4 rings
##    <chr>         <dbl>    <dbl>  <dbl>       <dbl>   <dbl>   <dbl>   <dbl> <dbl>
##  1 M             0.455    0.365  0.095       0.514  0.224   0.101    0.15     15
##  2 M             0.35     0.265  0.09        0.226  0.0995  0.0485   0.07      7
##  3 F             0.53     0.42   0.135       0.677  0.256   0.142    0.21      9
##  4 M             0.44     0.365  0.125       0.516  0.216   0.114    0.155    10
##  5 I             0.33     0.255  0.08        0.205  0.0895  0.0395   0.055     7
##  6 I             0.425    0.3    0.095       0.352  0.141   0.0775   0.12      8
##  7 F             0.53     0.415  0.15        0.778  0.237   0.142    0.33     20
##  8 F             0.545    0.425  0.125       0.768  0.294   0.150    0.26     16
##  9 M             0.475    0.37   0.125       0.509  0.216   0.112    0.165     9
## 10 F             0.55     0.44   0.15        0.894  0.314   0.151    0.32     19
## # ... with 4,167 more rows, and abbreviated variable names 1: whole_weight,
## #   2: shucked_weight, 3: viscera_weight, 4: shell_weight
\end{verbatim}

Question 1:

\begin{Shaded}
\begin{Highlighting}[]
\NormalTok{ages }\OtherTok{\textless{}{-}}\NormalTok{ abalone }\SpecialCharTok{\%\textgreater{}\%} 
  \FunctionTok{mutate}\NormalTok{(}\AttributeTok{age =}\NormalTok{ rings }\SpecialCharTok{+} \FloatTok{1.5}\NormalTok{)}
\FunctionTok{hist}\NormalTok{(ages}\SpecialCharTok{$}\NormalTok{age, }\AttributeTok{main =} \StringTok{"Histogram of Abalone Ages"}\NormalTok{, }
     \AttributeTok{xlab =} \StringTok{"Age"}\NormalTok{)}
\end{Highlighting}
\end{Shaded}

\includegraphics{hw2_files/figure-latex/question1-1.pdf}

\begin{Shaded}
\begin{Highlighting}[]
\NormalTok{ages}
\end{Highlighting}
\end{Shaded}

\begin{verbatim}
## # A tibble: 4,177 x 10
##    type  longest_sh~1 diame~2 height whole~3 shuck~4 visce~5 shell~6 rings   age
##    <chr>        <dbl>   <dbl>  <dbl>   <dbl>   <dbl>   <dbl>   <dbl> <dbl> <dbl>
##  1 M            0.455   0.365  0.095   0.514  0.224   0.101    0.15     15  16.5
##  2 M            0.35    0.265  0.09    0.226  0.0995  0.0485   0.07      7   8.5
##  3 F            0.53    0.42   0.135   0.677  0.256   0.142    0.21      9  10.5
##  4 M            0.44    0.365  0.125   0.516  0.216   0.114    0.155    10  11.5
##  5 I            0.33    0.255  0.08    0.205  0.0895  0.0395   0.055     7   8.5
##  6 I            0.425   0.3    0.095   0.352  0.141   0.0775   0.12      8   9.5
##  7 F            0.53    0.415  0.15    0.778  0.237   0.142    0.33     20  21.5
##  8 F            0.545   0.425  0.125   0.768  0.294   0.150    0.26     16  17.5
##  9 M            0.475   0.37   0.125   0.509  0.216   0.112    0.165     9  10.5
## 10 F            0.55    0.44   0.15    0.894  0.314   0.151    0.32     19  20.5
## # ... with 4,167 more rows, and abbreviated variable names 1: longest_shell,
## #   2: diameter, 3: whole_weight, 4: shucked_weight, 5: viscera_weight,
## #   6: shell_weight
\end{verbatim}

In this histogram, we see that the distribution of age is mostly
centered around 8 to 14 years old. There are very few abalone ages that
are older than 15 and there are extremely older than 25 years old. The
most common ages are between 10 and 12 years old as an unimodal,
right-skewed distribution.

Question 2

\begin{Shaded}
\begin{Highlighting}[]
\FunctionTok{set.seed}\NormalTok{(}\DecValTok{1500}\NormalTok{)}

\NormalTok{abalone\_split }\OtherTok{\textless{}{-}} \FunctionTok{initial\_split}\NormalTok{(ages, }\AttributeTok{prop =} \FloatTok{0.80}\NormalTok{, }
                               \AttributeTok{strata =}\NormalTok{ age)}

\NormalTok{train }\OtherTok{=} \FunctionTok{training}\NormalTok{(abalone\_split)}
\NormalTok{test }\OtherTok{=} \FunctionTok{testing}\NormalTok{(abalone\_split)}
\end{Highlighting}
\end{Shaded}

Question 3

\begin{Shaded}
\begin{Highlighting}[]
\NormalTok{abalone\_recipe }\OtherTok{\textless{}{-}} \FunctionTok{recipe}\NormalTok{(age }\SpecialCharTok{\textasciitilde{}}\NormalTok{ ., }\AttributeTok{data =}\NormalTok{ train) }\SpecialCharTok{\%\textgreater{}\%} 
  \FunctionTok{step\_rm}\NormalTok{(rings) }\SpecialCharTok{\%\textgreater{}\%}
  \FunctionTok{step\_dummy}\NormalTok{(}\FunctionTok{all\_nominal\_predictors}\NormalTok{()) }\SpecialCharTok{\%\textgreater{}\%}
  \FunctionTok{step\_interact}\NormalTok{(}\AttributeTok{terms =} \SpecialCharTok{\textasciitilde{}} \FunctionTok{starts\_with}\NormalTok{(}\StringTok{"type"}\NormalTok{)}\SpecialCharTok{:}\NormalTok{shucked\_weight }\SpecialCharTok{+}
\NormalTok{                  longest\_shell}\SpecialCharTok{:}\NormalTok{diameter }\SpecialCharTok{+}
\NormalTok{                  shucked\_weight}\SpecialCharTok{:}\NormalTok{shell\_weight) }\SpecialCharTok{\%\textgreater{}\%}
  \FunctionTok{step\_scale}\NormalTok{(}\FunctionTok{all\_numeric\_predictors}\NormalTok{()) }\SpecialCharTok{\%\textgreater{}\%}
  \FunctionTok{step\_center}\NormalTok{(}\FunctionTok{all\_numeric\_predictors}\NormalTok{())}
\NormalTok{abalone\_recipe}
\end{Highlighting}
\end{Shaded}

\begin{verbatim}
## Recipe
## 
## Inputs:
## 
##       role #variables
##    outcome          1
##  predictor          9
## 
## Operations:
## 
## Variables removed rings
## Dummy variables from all_nominal_predictors()
## Interactions with starts_with("type"):shucked_weight + longest_shell...
## Scaling for all_numeric_predictors()
## Centering for all_numeric_predictors()
\end{verbatim}

I shouldn't include rings to predict age since rings are directly
related to age as we would just add 1.5 years to rings and we would
automatically know the age of the abalone.

Question 4

\begin{Shaded}
\begin{Highlighting}[]
\NormalTok{lm\_model }\OtherTok{\textless{}{-}} \FunctionTok{linear\_reg}\NormalTok{() }\SpecialCharTok{\%\textgreater{}\%} 
  \FunctionTok{set\_engine}\NormalTok{(}\StringTok{"lm"}\NormalTok{)}
\NormalTok{lm\_model}
\end{Highlighting}
\end{Shaded}

\begin{verbatim}
## Linear Regression Model Specification (regression)
## 
## Computational engine: lm
\end{verbatim}

Question 5

\begin{Shaded}
\begin{Highlighting}[]
\NormalTok{lm\_wflow }\OtherTok{\textless{}{-}} \FunctionTok{workflow}\NormalTok{() }\SpecialCharTok{\%\textgreater{}\%} 
  \FunctionTok{add\_model}\NormalTok{(lm\_model) }\SpecialCharTok{\%\textgreater{}\%} 
  \FunctionTok{add\_recipe}\NormalTok{(abalone\_recipe)}

\NormalTok{lm\_fit }\OtherTok{\textless{}{-}} \FunctionTok{fit}\NormalTok{(lm\_wflow, train)}
\NormalTok{lm\_fit }\SpecialCharTok{\%\textgreater{}\%}
  \FunctionTok{extract\_fit\_parsnip}\NormalTok{() }\SpecialCharTok{\%\textgreater{}\%}
  \FunctionTok{tidy}\NormalTok{()}
\end{Highlighting}
\end{Shaded}

\begin{verbatim}
## # A tibble: 14 x 5
##    term                            estimate std.error  statistic  p.value
##    <chr>                              <dbl>     <dbl>      <dbl>    <dbl>
##  1 (Intercept)                   11.5          0.0373 307.       0       
##  2 longest_shell                  0.578        0.283    2.04     4.12e- 2
##  3 diameter                       2.20         0.307    7.17     9.43e-13
##  4 height                         0.207        0.0686   3.02     2.54e- 3
##  5 whole_weight                   5.02         0.388   12.9      2.51e-37
##  6 shucked_weight                -4.44         0.253  -17.5      7.22e-66
##  7 viscera_weight                -0.914        0.155   -5.89     4.34e- 9
##  8 shell_weight                   1.52         0.212    7.18     8.45e-13
##  9 type_I                        -0.951        0.117   -8.16     4.84e-16
## 10 type_M                        -0.283        0.105   -2.70     6.97e- 3
## 11 type_I_x_shucked_weight        0.531        0.0882   6.02     1.90e- 9
## 12 type_M_x_shucked_weight        0.294        0.112    2.64     8.41e- 3
## 13 longest_shell_x_diameter      -2.92         0.398   -7.34     2.61e-13
## 14 shucked_weight_x_shell_weight  0.0000418    0.204    0.000205 1.00e+ 0
\end{verbatim}

\begin{Shaded}
\begin{Highlighting}[]
\NormalTok{lm\_fit}
\end{Highlighting}
\end{Shaded}

\begin{verbatim}
## == Workflow [trained] ==========================================================
## Preprocessor: Recipe
## Model: linear_reg()
## 
## -- Preprocessor ----------------------------------------------------------------
## 5 Recipe Steps
## 
## * step_rm()
## * step_dummy()
## * step_interact()
## * step_scale()
## * step_center()
## 
## -- Model -----------------------------------------------------------------------
## 
## Call:
## stats::lm(formula = ..y ~ ., data = data)
## 
## Coefficients:
##                   (Intercept)                  longest_shell  
##                     1.145e+01                      5.779e-01  
##                      diameter                         height  
##                     2.204e+00                      2.071e-01  
##                  whole_weight                 shucked_weight  
##                     5.020e+00                     -4.442e+00  
##                viscera_weight                   shell_weight  
##                    -9.142e-01                      1.524e+00  
##                        type_I                         type_M  
##                    -9.512e-01                     -2.832e-01  
##       type_I_x_shucked_weight        type_M_x_shucked_weight  
##                     5.310e-01                      2.942e-01  
##      longest_shell_x_diameter  shucked_weight_x_shell_weight  
##                    -2.920e+00                      4.176e-05
\end{verbatim}

Question 6

\begin{Shaded}
\begin{Highlighting}[]
\NormalTok{gather\_data }\OtherTok{\textless{}{-}} \FunctionTok{data.frame}\NormalTok{(}\AttributeTok{type =} \StringTok{"F"}\NormalTok{, }\AttributeTok{longest\_shell =} \FloatTok{0.50}\NormalTok{, }\AttributeTok{diameter =} \FloatTok{0.10}\NormalTok{, }
                          \AttributeTok{height =} \FloatTok{0.30}\NormalTok{, }\AttributeTok{whole\_weight =} \DecValTok{4}\NormalTok{, }\AttributeTok{shucked\_weight =} \DecValTok{1}\NormalTok{, }
                          \AttributeTok{viscera\_weight =} \DecValTok{2}\NormalTok{, }\AttributeTok{shell\_weight =} \DecValTok{1}\NormalTok{, }\AttributeTok{rings =} \DecValTok{0}\NormalTok{)}

\FunctionTok{predict}\NormalTok{(lm\_fit, }\AttributeTok{new\_data =}\NormalTok{ gather\_data)}
\end{Highlighting}
\end{Shaded}

\begin{verbatim}
## # A tibble: 1 x 1
##   .pred
##   <dbl>
## 1  24.0
\end{verbatim}

Based on the information given, the abalone age is predicted to be
around 24 years old.

Question 7

\begin{Shaded}
\begin{Highlighting}[]
\NormalTok{multi\_metric }\OtherTok{\textless{}{-}} \FunctionTok{metric\_set}\NormalTok{(rsq, rmse, mae)}

\NormalTok{train\_res }\OtherTok{\textless{}{-}} \FunctionTok{predict}\NormalTok{(lm\_fit, }\AttributeTok{new\_data =}\NormalTok{ train }\SpecialCharTok{\%\textgreater{}\%} 
                       \FunctionTok{select}\NormalTok{(}\SpecialCharTok{{-}}\NormalTok{age))}

\NormalTok{train\_res }\OtherTok{\textless{}{-}} \FunctionTok{bind\_cols}\NormalTok{(train\_res, train }\SpecialCharTok{\%\textgreater{}\%} 
                         \FunctionTok{select}\NormalTok{(age))}

\FunctionTok{multi\_metric}\NormalTok{(train\_res, }\AttributeTok{truth =}\NormalTok{ age, }
                \AttributeTok{estimate =}\NormalTok{ .pred)}
\end{Highlighting}
\end{Shaded}

\begin{verbatim}
## # A tibble: 3 x 3
##   .metric .estimator .estimate
##   <chr>   <chr>          <dbl>
## 1 rsq     standard       0.551
## 2 rmse    standard       2.15 
## 3 mae     standard       1.55
\end{verbatim}

Based on the results, we see that the R squared value is around 0.551.
This means that there is a correlation (although not an extremely strong
one, but still a strong one) between our predicted values and response
values.

\end{document}
